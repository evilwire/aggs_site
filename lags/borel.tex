\begin{prop}[Ex. 17.8]\label{ex_17_8}
Let $G$ be a unipotent algebraic group acting on an affine variety 
$X$. Then all $G$-orbits are closed in $X$.
\end{prop}

\begin{proof}
Suppose for a contradiction that some orbit $Y$ of $G$ is not
closed, and let $\overline{Y}$ be its closure.

Consider the action of $G$ on the structure ring $K[X]$ by 
translation: $f \mapsto \lambda_x f$ where $\lambda_x f(y) = 
f(x^{-1}y)$ (see 8.5, 8.6). 

Let $I = I(\overline{Y})$, and let $J = I(\overline{Y} - Y)$.
Since $\overline{Y} - Y \neq \varnothing$, $(1) \neq J \subsetneq 
I$. We make two observations,

\begin{enumerate}
\item since $G$ acts on $Y$, $G$ acts on $\overline{Y}$ and 
$\overline{Y} - Y$ by Proposition 8.3.

\item since $G$ acts on $\overline{Y} - Y$ and $\overline{Y}$, 
then $G$ acts on $I$ and $J$. In particular, $G$ also acts on
$J/I$.
\end{enumerate}

By the proof of Proposition 8.6a), it is possible to find a
finite dimension vector subspace $V \subset J/I$ closed under 
$G$-action. In the case $G$ is unipotent, by Engel's Theorem
(Proposition 17.5), there exists some nonzero $\varphi \in I/J$ 
such that $\lambda_G \varphi = \varphi$.

Note that $\varphi$ represents some function $f$ on $K[X]$ for
which $f|_{\overline{Y}} \neq 0$. The fact that $G$ stabilizes 
$\varphi$ implies that $f|_Y$ is contant. Indeed, fix $y \in Y$;
then $Gy = Y$, and for $y' \in Y$, $y' = g^{-1}y$ for some $g \in G$.
Therefore, $f(y') = f(g^{-1}y) = \lambda_{g}f(y) = f(y)$.

However, $f$ constant on $Y$ implies $f$ constant on $\overline{Y}$.
That is $J = (1)$, a contradiction.
\end{proof}

In particular

\begin{cor}
Suppose $G$ is an unipotent algebraic group. All conjugacy 
classes of $G$ are closed.
\end{cor}
\begin{proof}
Let $G$ act on $G$ by conjugation, the later being affine.
\end{proof}

\begin{cor}
Connected unipotent algebraic groups have trivial character. 
(See 11.4)
\end{cor}
\begin{proof}
Let $G$ be an unipotent algebraic group, and let $\chi: G \to
\Gm$ be character. Consider
\[
\begin{diagram}
G \times \A^1 &\rTo{\phantom{aaa}\chi \times 1\phantom{aaa}} &\Gm 
\times \A^1 & \rTo{\phantom{aaaa}\mu\phantom{aaaa}} \A^1
\end{diagram}
\]
where $\mu: \Gm \times \A^1 \to \A^1$ is given by $(r,s) \mapsto 
rs$. This is a $G$ action on $\A^1$, and orbit of single nonzero
points are closed by Prop. \ref{ex_17_8}, and connected since $G$
is connected, hence $\A^1$ or a single point. The latter is impossible
since the $G$-action factors through $\Gm$. It follows that $G$ must
act trivially, whence $\chi$ must be trivial.
\end{proof}
